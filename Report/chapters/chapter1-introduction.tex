\chapter{Introduction}
% \label{chap:intro}
% \chaptermark{Optional running chapter heading}
% \section{Spacing \& Type}
% \label{sec:section}


% TODO: rewirite
% The rise of NFT-based games has introduced a paradigm shift in game mechanics,
% particularly in digital asset ownership and player-driven economies
% \cite{min_blockchain_2019}. However, despite the potential for innovation, there
% is noticeable stagnation in the development of novel game mechanics. Most
% existing games replicate established patterns, limiting the scope for deeper
% engagement and unique gameplay experiences \cite{jiang_cryptokitties_2021}. This
% gap highlights the need for new approaches to revitalize the blockchain gaming
% landscape.
%
% Emerging blockchain technologies, specifically the TON (The Open Network)
% blockchain, present a unique opportunity to address this challenge. TON’s
% architecture, designed for infinite scalability and advanced sharding,
% introduces a highly decentralized and resilient network. These features enhance
% transaction efficiency and enable developers to implement mechanics that were
% previously unfeasible on platforms like Ethereum due to scalability constraints
% and high transaction costs.
% Additionally, TON’s smart contract technology, while similar in purpose to
% Ethereum’s, offers significant structural and executional advantages, enabling
% more versatile game interactions\cite{durov_telegram_nodate}.
%
% This study adopts a three-pronged approach to explore how TON’s distinctive
% capabilities can enable innovative game mechanics and provide a path forward for
% the stagnating NFT gaming space:
% \begin{enumerate}
% 	\item An overview of TON’s architecture, focusing on its scalability,
% 	      sharding, and smart contract capabilities.
% 	\item A literature review of existing and potential game mechanics, with an
% 	      emphasis on those that leverage blockchain technologies.
% 	\item A practical demonstration showcasing secure and scalable mechanics
% 	      enabled by TON’s features, using a prototype or conceptual example.
% \end{enumerate}
%
% By bridging the gap in current research, this paper aims to provide both
% theoretical insights and practical guidance for leveraging TON in blockchain
% gaming. The broader implications of this research include improving user
% engagement, fostering novel gameplay experiences, and setting the stage for
% sustainable player-driven economies. By combining an analysis of TON’s technical
% advantages with concrete examples, this study seeks to inspire developers and
% researchers to explore untapped possibilities within this dynamic and evolving
% space.
