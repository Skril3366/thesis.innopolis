\chapter{Literature Review}

\section{NFT-based Game Mechanics}

The advent of NFT-based game mechanics has revolutionized the digital gaming
landscape by introducing novel paradigms that redefine asset ownership, player
incentives, and interactive experiences. Central to this transformation is the
concept of verifiable asset ownership, where blockchain technology ensures that
digital assets are uniquely identifiable and securely traceable, challenging
traditional centralized models. The play-to-earn reward systems further bridge
virtual economies with real-world finance, offering players tangible financial
incentives for their in-game achievements. Additionally, NFT breeding and
fusion, along with asset staking and yield generation, introduce complex
mechanisms for asset creation and passive income, respectively, while NFT
crafting and resource transformation leverage blockchain's principles of digital
scarcity. Rental and delegation systems enhance asset liquidity and inclusivity,
and cross-game asset utilization extends the utility of NFTs across multiple
platforms. Tokenized governance participation empowers players with
decision-making capabilities, fostering a decentralized control of game
evolution. Dynamic NFT evolution and fractional ownership systems further enrich
the gaming experience by allowing assets to adapt over time and democratizing
investment opportunities. Collectively, these mechanics not only enhance player
engagement but also open new avenues for innovative monetization strategies,
underscoring the potential and complexity inherent in NFT-based gaming.

\subsection{Verifiable Asset Ownership}

Blockchain technology has transformed digital gaming by introducing verifiable
asset ownership a marked departure from traditional centralized models.
Conventional online games typically vest control of virtual assets in the game
operator, which limits players’ rights regarding asset transfer and
authenticity. In contrast, blockchain-based systems leverage decentralized
ledgers and smart contracts to certify ownership, ensuring that digital game
assets (such as NFTs) are uniquely identifiable and securely traceable
\cite{minBlockchainGamesSurvey2019}. Scholars have increasingly focused on the
technical and economic implications of this shift. The integration of
non-fungible tokens and immutable smart contracts offers a robust method for
verifying asset authenticity and provenance
\cite{gaoEmpiricalStudyAdoption2021}. This verifiability is achieved through a
combination of cryptographic techniques and decentralized storage, ensuring that
an asset’s recorded ownership cannot be altered without detection. Nevertheless,
challenges regarding scalability, high transaction fees, and latency have been
identified as key obstacles to real-time asset verification
\cite{rishiwalBlockchainSecureGamingEnvironments2024}. Moreover, although the
decentralized framework eliminates the need for trusted intermediaries,
questions remain about enforcing digital property rights in cross-platform
environments and sustaining such economic models
\cite{delfabbroUnderstandingMechanicsConsumer2022}. The theoretical
underpinnings here that inscribing ownership directly on the blockchain not only
enhances security but also introduces liquidity to previously illiquid digital
assets challenge traditional approaches to asset management and open new avenues
for innovative monetization strategies \cite{kocerEffectsBlockchainGame2022}.
Overall, these insights establish verifiable ownership as a critical component
for understanding subsequent blockchain innovations in gaming.

\subsection{Play-to-Earn Reward Systems}

The rapid evolution of blockchain-based gaming has fostered a paradigm in which
financial incentives motivate gameplay and bridge the gap between virtual
economies and real-world finance. Central to this innovation is the design of
reward systems that convert in-game achievements into tangible assets, enabling
players to earn tokens, NFTs, or other digital commodities convertible into fiat
currency. A core tenet of the play-to-earn (P2E) model is to compensate players
for the time and effort they invest, thereby addressing limitations of
traditional gaming models where rewards are limited solely to in-game benefits
\cite{delfabbroUnderstandingMechanicsConsumer2022}. These systems facilitate an
open market in which earned tokens and assets can be traded or staked, linking
player performance with financial rewards. Blockchain integration further
ensures that reward distributions are secure, transparent, and resistant to
tampering \cite{duguleanaEmergingTrendsPlaytoEarn2024,
	shazhaevPlaytoHashEconomicsMetaverses2022}. However, an overemphasis on
extrinsic financial incentives can diminish intrinsic motivators such as
personal achievement and enjoyment. Research indicates that prioritizing
monetary gain may transform recreational gameplay into a high-pressure economic
endeavor \cite{delicProfilingPotentialRisks2024}, while the speculative nature
of asset trading may introduce volatility, undermining long-term engagement
\cite{leeExaminingConsumerMotivations2023,
	delfabbroUnderstandingMechanicsConsumer2022}. Hence, effective P2E systems must
carefully balance financial and intrinsic rewards to ensure sustainable and
engaging gameplay.

\subsection{NFT Breeding and Fusion}

In the evolving landscape of digital asset management, NFT breeding and fusion
have emerged as innovative mechanisms for generating, combining, and enhancing
unique tokens. Originating in crypto-games, these processes are central both to
the creation of novel virtual products and to broader discussions on economic
and cultural impacts within blockchain ecosystems. Early
implementations exemplified by projects such as CryptoKitties demonstrated that
chance-based breeding mechanics can create NFTs with variable traits and market
value, similar to loot boxes and gacha games
\cite{oliverjamesscholtenEthereumCryptoGamesMechanics2019}. In addition, fusion
techniques that consolidate multiple tokens into a single enhanced NFT add
complexity by blending elements of randomness with engineered asset evolution.
Critical analysis shows that NFT breeding typically relies on probabilistic
trait inheritance, thereby incentivizing continued user engagement through the
possibility of achieving rare characteristics
\cite{oliverjamesscholtenEthereumCryptoGamesMechanics2019}. In parallel, fusion
processes allow for the combination of distinct NFTs to produce composite tokens
with augmented features for example, as seen in Nike’s CryptoKicks initiative
\cite{volodevaBlockchainFashionIndustry2024}. Yet, economic constraints such as
the arbitrage constraint affecting breeding outcomes highlight the necessity for
careful balance between game tokens and market costs
\cite{meisterNFTsGameAfoot2022}. From an implementation perspective,
blockchain-based smart contracts enforce randomness, traceability, and
non-fungibility, although regulatory uncertainties, market volatility, and fair
distribution challenges persist. These discussions underscore how NFT breeding
and fusion exemplify both the potential and the complexity inherent in
innovative digital asset creation.

\subsection{Asset Staking and Yield Generation}

Asset staking and yield generation have emerged as pivotal mechanisms in
blockchain-based gaming by offering players an alternative revenue stream. By
committing digital assets such as tokens or NFTs to smart contracts in exchange
for passive rewards, players are enabled to forgo immediate in-game utility in
favor of long-term yield. Research on token circulation demonstrates that
staking can regulate in-game economies and mitigate market volatility
\cite{yuSoKPlaytoEarnProjects2022a}. Complementary studies further examine the
psychological benefits of sustained passive rewards, suggesting that such
incentives foster increased player engagement over extended periods
\cite{duguleanaEmergingTrendsPlaytoEarn2024}. However, staking involves
trade-offs; removing assets from active gameplay may reduce immediate
interaction, and mandatory lock-up periods can create barriers for new players
as well as potential economic imbalances
\cite{meisterbernhardkNFTsGameAfoot2022}. Hence, a balanced staking mechanism
must provide attractive passive income while preserving active gameplay
dynamics.

\subsection{NFT Crafting and Resource Transformation}

NFT crafting introduces a novel game mechanic in which players combine in-game
resources or lower-value NFTs to create new, higher-value digital assets. This
process distinguishes itself from traditional crafting systems by incorporating
blockchain’s principles of digital scarcity and permanence for instance,
component NFTs are often permanently transformed or destroyed during crafting
\cite{shiApplicationsNFTsAI2024}. Fixed or discoverable crafting recipes
encourage strategic resource gathering, deepening player engagement and
exploration, while probabilistic outcomes add an essential element of risk and
reward. At the same time, the complexity of balancing resource costs with
resultant rewards can lead to economic inflation if not managed properly
\cite{wangNonFungibleTokenNFT2021}. Additional technical obstacles, such as
slow transaction confirmations and high gas fees, further complicate the user
experience and necessitate a robust interface design that accommodates NFTs both
as collectibles and interactive game components
\cite{murrayMechanicsBlockchainTaxonomy2022, popescuNonFungibleTokensNFT2021}.
Thus, NFT crafting encapsulates both the promise and the challenges of merging
traditional resource transformation with digital scarcity, a synthesis that is
critical for the development of secure, innovative NFT-based games.

\subsection{Rental and Delegation Systems}

Recent advancements in blockchain gaming have highlighted the importance of
rental and delegation systems as innovative approaches to digital asset
management. These systems enable NFT owners to temporarily grant usage rights
while maintaining overall ownership, thereby creating opportunities for passive
income and expanding access to premium in-game assets. The underlying principle
is one of efficient asset utilization; by monetizing NFTs without transferring
full ownership, these systems enhance liquidity and inclusivity
\cite{popescuNonFungibleTokensNFT2021,
	shazhaevPlaytoHashEconomicsMetaverses2022}. Automated smart contracts underpin
these mechanisms by enforcing time-limited agreements, revenue-sharing
protocols, and safeguards against misuse
\cite{stamatakisBlockchainPoweredGamingBridging2024}. Empirical examples such as
scholar-manager programs observed in games like Axie Infinity demonstrate how
lending arrangements can generate shared revenue
\cite{tanAssetizingVideoGame2025}. Nonetheless, challenges including contractual
exploitation \cite{delfabbroUnderstandingMechanicsConsumer2022}, asset theft,
and market volatility necessitate robust contract design and governance
frameworks to ensure system stability.

\subsection{Cross-Game Asset Utilization}

Cross-game asset utilization represents a transformative mechanism by which NFTs
retain their identity and value across multiple digital environments. Through
blockchain-verified credentials, an asset can function beyond the confines of a
single game \cite{popescuNonFungibleTokensNFT2021}. This mechanism relies on
smart contracts and shared metadata standards to ensure uniform recognition
across diverse platforms. Initiatives like the “Nifty License” exemplify how
assets from one game may be integrated into others such as KittyRace,
KittyBattle, and KotoWars \cite{minBlockchainGamesSurvey2019}. By extending
asset utility across games, this approach supports economic growth through
cross-game marketplaces that promote liquidity and bolster player retention
\cite{choiStudyElementsBusiness2021}. Despite its advantages, significant
challenges remain regarding technical complexity, economic balance within game
environments, and regulatory compliance \cite{shiApplicationsNFTsAI2024}. These
issues underscore the need for further standardization and innovation in
achieving true NFT interoperability.

\subsection{Tokenized Governance Participation}

Tokenized governance has become a transformative mechanism in NFT-based games,
enabling decentralized control of game evolution through blockchain-based voting
systems. By endowing players with governance tokens or NFTs, these systems allow
for direct influence over game rules, economic models, and content updates,
thereby aligning player interests with long-term development objectives
\cite{muhammadDeterminantsAINonFungible2023}. Various models for distributing
voting rights exist; some rely on direct token holdings while others incorporate
reputation scores to prevent power concentration
\cite{chenFrameworkBasedDAO2024}. Decentralized Autonomous Organizations (DAOs)
facilitate community-driven proposals that shift control from developers to a
distributed stakeholder network \cite{shazhaevPlaytoHashEconomicsMetaverses2022,
	yuSoKPlaytoEarnProjects2022a}. Nevertheless, challenges such as voter apathy and
plutocratic imbalances where large token holders can dominate outcomes have been
identified \cite{yuSoKPlaytoEarnProjects2022a, conlonProblemNFTs2023}. Emerging
solutions, including decay mechanisms for reputation scores and delegated
voting, are being explored to address these issues
\cite{chenFrameworkBasedDAO2024, stamatakisBlockchainPoweredGamingBridging2024}.
Overall, tokenized governance represents an important evolution in democratizing
decision-making in game development while incentivizing sustained stakeholder
engagement.

\subsection{Dynamic NFT Evolution}

Dynamic NFT evolution extends the concept of static digital ownership by
allowing assets to adapt and transform over time in response to gameplay events
and external data. This mechanism leverages sophisticated smart contract designs
to update an NFT’s metadata in real time, ensuring that every change is securely
recorded on the blockchain \cite{barbaraguidiNFT10NFT2023}. Notable implementations include games where characters or in-game items
evolve through level ups or trait changes triggered by user interactions or
on-chain events  \cite{oliverjamesscholtenEthereumCryptoGamesMechanics2019,
	patelBlockchainGamingBuilding2023}. However, the increased complexity introduced by
dynamic evolution requires rigorous security testing and careful calibration to
avoid gameplay imbalances, particularly when external oracles are involved.
Despite these challenges, dynamic NFT evolution offers a promising avenue for
enhancing interactivity and enriching the overall gaming experience.

\subsection{Fractional Ownership Systems}

Fractional ownership systems have emerged as a transformative paradigm in
digital asset management by leveraging blockchain technology to democratize
investment, enhance liquidity, and broaden market participation. By dividing
high-value assets into smaller, tradeable tokens, these systems lower entry
barriers and enable secure, verifiable transactions through smart contracts
\cite{popescuNonFungibleTokensNFT2021, michiNFTArtRestrained2022}. Each
transaction is immutably recorded on the blockchain, thereby mitigating
traditional trust issues while facilitating risk-sharing among investors
\cite{duguleanaEmergingTrendsPlaytoEarn2024, barbaraguidiNFT10NFT2023}. Fractional ownership has found applications in diverse fields including
art, real estate, and decentralized finance, demonstrating its potential to
support diversified investment strategies \cite{michiNFTArtRestrained2022,
	marinReviewBlockchainTokens2023}. In this context,
fractional ownership aligns with the broader objectives of this study by
exploring secure, scalable economic models within NFT-based gaming.


\section{Comparative analysis of TON's and Ethereum's Features}

This section examines key technical differences between TON and Ethereum that
are critical for developing NFT-based game mechanics. By comparing aspects such
as smart contract modifiability, the atomicity of contract calls, scalability
architecture, and off-chain asset handling, this review highlights how each
platform’s design choices affect error management, adaptability, and
performance. These insights set the stage for understanding which blockchain
features best support innovative game development.

\subsection{Smart Contract Modifiability}

Blockchain platforms differ significantly in their approach to modifying smart
contracts. TON offers native upgrade mechanisms that enable developers to alter
contract logic post-deployment through built-in update functions
\cite{songEnhancingOpenNetwork2025}. This flexibility facilitates rapid
iteration and dynamic enhancements while minimizing downtime. In contrast,
Ethereum enforces immutability to ensure stability and trust but requires the
use of complex upgrade patterns—such as proxy contracts—to implement changes
\cite{pathakInteroperabilityLegalInterpretation2024,
	caiDecentralizedApplicationsBlockchainEmpowered2018}. Thus, while TON
prioritizes adaptability through post-deployment modifications, Ethereum
emphasizes security integrity at the expense of responsiveness to change.

\subsection{Atomicity in Smart Contract Calls}

The paradigms for message structure and error handling diverge markedly between
the two platforms. TON employs a cell-based data structure in which all data,
including messages, are stored in cells capable of holding up to 1023 bits and
referencing up to four other cells  \cite{durovTelegramOpenNetwork2020}.
Organized as directed acyclic graphs to prevent cyclic dependencies, this
mechanism requires strict synchronization when serializing messages according to
TL-B schemes ( \cite{durovTelegramOpenNetwork2020} ). The primary processing
function, recv\_internal, explicitly manages errors by bouncing messages that
fail processing, thereby obliging developers to address issues such as the
“Unchecked Bounced Message” defect \cite{songEnhancingOpenNetwork2025}.

By contrast, Ethereum utilizes a conventional message model within its virtual
machine. Here, messages are virtual objects encapsulating the sender, recipient,
ether amount, data, and a STARTGAS value, generated via the CALL opcode
\cite{EthereumWhitepaper2025}. Ethereum’s execution model is inherently
atomic—if any component of a transaction encounters an error, the entire
sequence is reverted, ensuring complete transactional integrity
\cite{buterinEthereumNextGenerationSmart2014}. This all-or-nothing approach
simplifies failure recovery, whereas TON’s non-atomic call structure means that
state changes made by preceding calls are not automatically rolled back on
failure \cite{songEnhancingOpenNetwork2025}. In addition, TON’s reliance on
builder and slice primitives for data handling demands careful synchronization
to avoid critical errors \cite{songEnhancingOpenNetwork2025}, while Ethereum’s
established virtual machine abstracts such complexities
\cite{EthereumWhitepaper2025}. Together, these differences necessitate that
developers working on TON implement more defensive programming strategies and
explicit recovery protocols compared to the straightforward atomic transactions
of Ethereum.

\subsection{Scalability Architecture}

Scalability is a fundamental challenge with significant implications for
transaction throughput and overall user experience. TON adopts an innovative
2D-blockchain structure comprising a masterchain, multiple workchains, and
numerous shardchains. In this configuration, the masterchain maintains protocol
and validator data, workchains handle transactions and smart contracts under
varied rules, and shardchains enable dynamic sharding based on account
distribution . Although current throughput may be modest, TON’s architecture
promises significant scalability potential as user engagement increases
\cite{durovTelegramOpenNetwork2020}.

In contrast, Ethereum is constrained by its design, reliably processing around
20 transactions per second in practice despite theoretical capacities of up to
1,000 TPS. Extended consensus times and continuous growth in blockchain size
exacerbate these limitations \cite{khanSystematicLiteratureReview2021}. These constraints contribute to high gas fees and network
congestion—as seen during the CryptoKitties surge
\cite{caiDecentralizedApplicationsBlockchainEmpowered2018, jiangCryptoKittiesTransactionNetwork2021} which
restrict game design to simpler, turn-based mechanisms. Although TON’s dynamic
sharding and lower transaction costs offer greater potential for intricate game
mechanics, developers must remain mindful of challenges such as its non-atomic
call behavior \cite{songEnhancingOpenNetwork2025}. Overall, the
scalability comparison underscores the trade-offs between Ethereum’s established
ecosystem and TON’s promising yet nuanced infrastructure.

\subsection{Off-Chain Asset Handling}

Managing off-chain assets is crucial for blockchain game development,
particularly for preserving data integrity while minimizing on-chain storage
costs. On Ethereum, key components like metadata, images, and state data are
typically stored off chain with only a cryptographic hash or pointer recorded on
the blockchain \cite{patelBlockchainGamingBuilding2023,
	arifBlockchainBasedMultiplayerTransaction2020}. This approach maintains
verifiable ownership but can impose rigidity; modifications to off-chain asset
references often require complex proxy patterns or complete redeployments.

TON addresses these challenges by supporting native smart contract
upgradability, which allows for rapid updates to off-chain asset references
without resorting to intricate workarounds \cite{songEnhancingOpenNetwork2025}.
When combined with its 2D blockchain architecture, TON is well positioned to
manage high volumes of transactions. Furthermore, its non-atomic call mechanism
permits incremental updates by enabling partial transaction commitment in the
event of processing errors \cite{songEnhancingOpenNetwork2025}. In this way,
TON’s flexible approach to error handling and system updates contrasts with
Ethereum’s all-or-nothing model, offering enhanced adaptability for dynamic
off-chain asset management.


% TODO:
% \section{Security Challenges in Smart Contract Development}
% \section{The Need For a Novel Approach}
